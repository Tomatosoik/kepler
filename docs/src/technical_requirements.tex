\documentclass[a4paper, 12pt]{article}

\usepackage[english,russian]{babel}
\usepackage[T2A]{fontenc}
\usepackage[utf8]{inputenc}

\usepackage{indentfirst}
\usepackage{geometry}
\geometry{left=30mm, right=15mm, top=20mm, bottom=20mm}

\usepackage{hyperref}


\title{Техническое задание для ``Задача Кеплера''}
\author{Автор: Никита Томашевский \thanks{При участии Павла Воронова ФКТиПМ}}
\date{\today}

\begin{document}
 \maketitle\newpage
 \tableofcontents\newpage


 \section{Введение}
 \subsection{Наименование}
 Моделирование задачи Кеплера о двух тел в двух случаях:
 \begin{enumerate}
  \item Первая материальная точка вращается вокруг второй материальной точки по эллиптической орбите.
  \item Первая материальная точка пролетает рядом со второй материальной точкой.
 \end{enumerate}
 \subsection{Краткая характеристика области применения программы}
 Программа, моделирующая задачу Кеплера о двух тел в двух частных случаях, может использоваться в астрономии, например, в небесной механике, для моделирования траектории движения двух космических объектов. Может применяться в учебных целях для визуализации поставленной задачи и демонстрации законов Кеплера.


 \section{Основания для разработки}
 \subsection{Документ}
 ПР-МОД-090302.19.2023 ТЗ 01-ЛУ

 Утверждён руководителем Н. Н. Куликовой 09.04.2023
 \subsection{Наименование}
 Моделирование задачи Кеплера о двух тел в двух случаях:
 \begin{enumerate}
  \item Первая материальная точка вращается вокруг второй материальной точки по эллиптической орбите.
  \item Первая материальная точка пролетает рядом со второй материальной точкой.
 \end{enumerate}
 \section{Назначение разработки}
 \subsection{Функциональное и эксплуатационное назначение программы}
 Функциональное назначение программы заключается в моделировании задачи Кеплера о двух тел в двух частных случаях, а также расчёте различных параметров, например, траектории движения, скорости.
% ========================= ПАТОМ ГНИДА ДОПИШИ
 Эксплуатационное назначение программы заключается в её использовании при моделировании задачи Кеплера о двух тел. Она может быть полезна в небесной механике для моделирования траектории движения двух космических объектов, в учебных целях для изучении законов Кеплера.


 \section{Требования к программе}
 \subsection{Требования к функциональным характеристикам}

 \subsubsection{Требования к составу выполняемых функций:}

 \begin{enumerate}
  \item Расчёт траектории движения первой материальной точки относительно второй.
  \item Расчёт скорости движения первой материальной точки.
% =========================  ПАТОМ ГНИДА ДОПИШИ
  \item Возможность задания начальных условий для материальных точек, например, масса, стартовая скорость.
% =========================  ПАТОМ ГНИДА ДОПИШИ
  \item Предоставление результатов расчетов в удобной форме.
  \item Обеспечение точности расчетов и устойчивости работы программы при различных входных данных.
  \item Интуитивно понятный интерфейс пользователя для удобного ввода данных и просмотра результатов расчетов.
 \end{enumerate}

 \subsubsection{Требования к организации входных и выходных данных:}

 \begin{enumerate}
  \item Входные данные должны содержать значения начальных параметров заряда, таких как масса и начальная скорость.
% =========================  ПАТОМ ГНИДА ДОПИШИ
  \item Программа должна проверять правильность введенных данных и сообщать об ошибках в случае некорректных значений.
  \item Результаты расчетов должны предоставляться в удобной форме, например, в виде графиков или таблиц.
  \item Программа должна быть устойчива при различных входных данных и обеспечивать высокую точность расчетов.
  \item Интерфейс пользователя должен быть интуитивно понятным и удобным для ввода данных и просмотра результатов. расчетов.
 \end{enumerate}

 \subsubsection{Требования к временным характеристикам:}

 \begin{enumerate}
  \item Программа должна обеспечивать высокую скорость расчетов для быстрого моделирования движения зарядов в реального времени.
  \item Результаты расчетов должны предоставляться немедленно после ввода начальных параметров и изменения параметров полей.
  \item Программа должна быть устойчива при длительном использовании и не вызывать перегрев компьютера или других проблем с производительностью.
 \end{enumerate}

 \subsection{Требования к надёжности}
 \subsubsection{Требования к обеспечению надёжного функционирования:}
 \begin{enumerate}
  \item Программа должна быть оптимизирована для работы на различных операционных системах и аппаратных платформах.
  \item Программа должна использовать эффективный алгоритм расчетов, который не будет вызывать перегрузку процессора или использовать слишком большой объем оперативной памяти.
  \item Программа должна иметь возможность оптимизации использования ресурсов компьютера, чтобы избежать перегрева или других проблем с производительностью.
  \item Программа должна обеспечивать проверку корректности входных данных, чтобы избежать ошибок при расчетах.
 \end{enumerate}


% \subsection{Условия эксплуатации}
 \subsection{Требования к составу и параметрам технических средств}
% <=============================================================== ПАША ВИ НИД ТО Ю ДОДЕЛАЙ БЛИНБ

 \subsection{Требования к информационной и программной совместимости}
 \subsubsection{Требования к данным}
 \begin{enumerate}
  \item Входные данные должны содержать информацию о массе материальных точек и стартовой скорости первой точки.
% =========================  ПАТОМ ГНИДА ДОПИШИ
  \item Выходные данные должны содержать информацию о траектории движения материальных точек и конечных скоростях.
% =========================  ПАТОМ ГНИДА ДОПИШИ
  \item Для расчета траектории и скорости должны использоваться соответствующие формулы и уравнения.
% =========================  ПАТОМ ГНИДА ДОПИШИ
 \end{enumerate}

 \subsubsection{ Требования к исходным кодам, языкам программирования и программным средствам}
 \begin{enumerate}
  \item Программа должна быть написана на языке программирования C++.
  \item Для численного интегрирования можно использовать стандартные библиотеки языка программирования или специализированные библиотеки, такие, как math.h.
% <=============================================================== ПАША ВИ НИД ТО Ю ДОДЕЛАЙ БЛИНБ
 \end{enumerate}


 \subsubsection{Требования к техническим средствам}
% <=============================================================== ПАША ВИ НИД ТО Ю ДОДЕЛАЙ БЛИНБ
 \subsubsection{Требования к маркировке и упаковке}
% <=============================================================== ПАША ВИ НИД ТО Ю ДОДЕЛАЙ БЛИНБ
 \subsection{Требования к транспортированию и хранению}
 \begin{itemize}
  \item Требования к надежности:
  \item Требования к целостности;
  \item Требования к удобству:
% <=============================================================== ПАША ВИ НИД ТО Ю ДОДЕЛАЙ БЛИНБ
 \end{itemize}

 \subsection{Требования к программной документации}

 \subsubsection{Предварительный состав программной документации}
 \begin{enumerate}
  \item Техническое задание: документ, описывающий требования к программе от заказчика.
  \item Алгоритм: документ, описывающий правила создания программы
  \item Код: сам код программы, который может быть документирован комментариями.
  \item Руководство по разработке: документ, описывающий процесс разработки программы, рекомендации и правила для разработчиков.
 \end{enumerate}

 \subsubsection{Специальные требования к программной документации}
 \begin{enumerate}
  \item Обязательное наличие комментариев в коде программы.
  \item Использование стандартов и методологий разработки программного обеспечен
 \end{enumerate}


 \section{Технико-экономические показатели}
 \subsection{Ориентировочная эффективность}
 Ориентировочная эффективность данной программы зависит от многих факторов: объём данных для расчетов, используемые алгоритмы и технологии, сложность непосредственно самой модели. Конкретная ориентировочная эффективность программы может быть определена только после ее разработки и тестирования.

 \subsection{Предполагаемая годовая потребность}
 Предполагаемая годовая потребность такой программы зависит от многих факторов: сложность проектов, в которых используется моделирование задачи Кеплера о двух тел, количество и требования пользователей. Однако при использовании в учебных целях годовая потребность предполагается минимальной.

 \subsection{Экономические преимущества разработки по сравнению с аналогами}
 \begin{enumerate}
  \item Снижение затрат на разработку и производство изделий и оборудования, которые используются для моделирования задачи Кеплера о двух тел.
  \item Увеличение точности и скорости вычислений.
  \item Увеличение эффективности процессов проектирования и разработки новых изделий и технологий, что позволяет сократить время и затраты на их создание.
  \item Улучшение качества продукции за счёт точного моделирования задачи Кеплера о двух тел.
 \end{enumerate}

 \subsection{Стадии и этапы разработки программы}
 \begin{enumerate}
  \item Исследовательская стадия:
   \begin{itemize}
    \item[—] Определение требований к программе;
    \item[—] Изучение существующих решений и алгоритмов;
    \item[—] Анализ возможных методов моделирования задачи Кеплера о двух тел в двух частных случаях.
   \end{itemize}
     \item Проектировочная стадия:
   \begin{itemize}
    \item[—] Разработка общей структуры программы;
    \item[—] Выбор движков, библиотек для написания программы;
    \item[—] Создание архитектуры программы и определение ее модулей.
   \end{itemize}
     \item Реализационная стадия:
   \begin{itemize}
    \item[—] Написание кода программы;
    \item[—] Тестирование отдельных модулей и программы в целом;
    \item[—] Отладка и исправление ошибок.
   \end{itemize}
 \end{enumerate}



 \section{Порядок контроля и приёмки}
 \begin{enumerate}
  \item Тестирование на различных начальных условиях, чтобы убедиться в правильности расчетов для всех возможных сценариев.
  \item Проверка точности расчетов и сравнение результатов с известными аналитическими решениями.
  \item Проверка на устойчивость и корректность работы программы при изменении параметров системы.
  \item Проверка на соответствие требованиям производительности и скорости расчетов.
  \item Проверка на наличие ошибок и их исправление.
  \item Проверка на соответствие всем требованиям.
 \end{enumerate}

\end{document}
